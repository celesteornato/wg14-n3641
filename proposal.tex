% Created 2025-07-07 Mon 01:01
% Intended LaTeX compiler: pdflatex
\documentclass[a4paper, 12pt]{article}
\usepackage[utf8]{inputenc}
\usepackage[T1]{fontenc}
\usepackage{graphicx}
\usepackage{longtable}
\usepackage{wrapfig}
\usepackage{rotating}
\usepackage[normalem]{ulem}
\usepackage{amsmath}
\usepackage{amssymb}
\usepackage{capt-of}
\usepackage{hyperref}
\makeatletter \@ifpackageloaded{geometry}{\geometry{margin=2.5cm}}{\usepackage[margin=2.5cm]{geometry}} \makeatother
\parskip=12pt plus 1pt
\author{Céleste Ornato <celeste@ornato.com>}
\date{\today}
\title{Declaration-level static assertions\\\medskip
\large WG14 N3641}
\hypersetup{
 pdfauthor={Céleste Ornato <celeste@ornato.com>},
 pdftitle={Declaration-level static assertions},
 pdfkeywords={},
 pdfsubject={},
 pdfcreator={Emacs 30.1 (Org mode 9.7.31)}, 
 pdflang={English}}
\usepackage{biblatex}

\begin{document}

\maketitle
\begin{itemize}
\item \textbf{Proposal category}: Feature request
\item \textbf{Target audience}: Developers writing libraries
\end{itemize}
\section{Abstract}
\label{sec:orge50960a}
This proposal aims to add the \texttt{[[assume(expr)]]} and \texttt{[[assume(expr, msg)]]}
attributes to function declarations, allowing the designer of a library to specify
statically-verifiable constraints in function parameters.

This feature allows both for added safety on the caller side, and possibly for
further optimization on the compiler side.
\section{Foreword}
\label{sec:orga5c4b6f}
I have, between the initial writing of this paper and its submission, been made aware
of the fact that C++26 has implemented a similar feature, through ``function contract
specifiers''.

While similar in the abstract, the proposed feature is fairly different in practice:
For one, It aims to add compile-time security, rather than at execution-time.
It also does not add any new syntax, insofar as functions compiled with this feature
should still be callable from older code without issues\footnote{This is only guaranteed if the caller adheres to the guidelines set by the library
developer regarding proper usage of the function; see \ref{sec:orgfa252a4}.}.

I believe it is still important to define this feature independently from other standards,
so as to better suit the needs and aims of C developers.
\section{Introduction}
\label{sec:orgb625cb3}
Inattention and forgetfulness cause developers to write insecure code.

If one wishes to convey to the user of a library that certain function parameters
can cause undefined or unwanted behavior, they can only do so in written
documentation.  This leaves the developer in a precarious position: they can
either add checks at execution-time, which come with an unwanted performance
overhead and having to set up and document an error code system,
or they can assume that everyone will read the documentation, which may create
vulnerabilities.

This proposal seeks to make some of these checks possible without overhead, by implicitly
making static assertions before the function call.

This feature can be seen as an extension to the C99 ``static array index in
function parameters'' feature, in that it both allows for optimizations by the
compiler and for more compile-time checking for the user.
\section{Proposal}
\label{sec:org7e11dbf}
\subsection{Technical Description}
\label{sec:org5227bef}
When the attribute \texttt{[[assume(expr)]]} is associated with the declaration of a
function \texttt{foo}, any call to \texttt{foo} will require the compiler to check that \texttt{expr}
is \textbf{not provably false} (see \ref{sec:org7a60b83}) given the parameters.
When compiling \texttt{foo}, it is assumed that \texttt{expr} is always true.  Calling the
function with parameters that do not respect \texttt{expr} is undefined behavior.

\texttt{expr} may address the function parameters by their name, or by calling them
\texttt{\$n}, where \texttt{n} is the 0-indexed number of the parameter from left to right.

If a referenced variable is an array, it should decay into a pointer.  While it
could have allowed for more features, this would have created bug-prone behavior
depending on whether the author of \texttt{expr} expected a pointer or an array.  In any
case, most problems related to the size of arrays are already solved by having
static indices in the function signature.
\subsection{Rationale}
\label{sec:orgfa252a4}
Introducing more undefined behavior in C2y may be controversial; it would certainly
not be seen as ``Enabling secure programming'' at a first glance.

This feature is meant for cases where the developer already considers certain
parameters to be ``Undefined behavior''.  At this point, it does not matter
whether the standard considers the code to be defined, because the results would
still be unexpected and prone to breakage.

Allowing for further optimisation is just a welcome consequence of one being able
to specify what they consider undefined behavior.  In reality, the wanted feature is the compiler
being capable of detecting mistakes.  Function signatures using this attribute also
make the code self-documenting, which makes it easier to understand the assumptions made by the developer
when writing the function.
\subsection{Example}
\label{sec:org6e883ca}
\begin{verbatim}
[[assume($1 != 0)]]
int division(int, int);

[[assume(a >= b, "Substraction requires a >= b")]]
unsigned int subtract(unsigned int a, unsigned int b);

int main(void)
{
    // These compile, with no execution-time overhead.
    int result1 = division(9, 3);
    unsigned int result2 = subtract(200, 60);

    // Error: "Assumption '$1 != 0' is false."
    int result3 = division(2, 0);
    // Error: "Substraction requires a >= b."
    unsigned int result4 = subtract(2, 9);
}
\end{verbatim}
\subsection{Quirks}
\label{sec:org7a60b83}
\textbf{An assumption whose validity cannot be proven will be treated as valid.}
This should not be a problem, as this would just mean coming back to the
status quo of having to be careful as a user.
This aligns with how static array indices work in function signatures.

Due to the way static assertions work in general, we cannot always assert
the validity of every expression.  However, in ``safer'', more restricted
styles of C, where we only work with automatic-storage-duration variables
and constant-sized arrays, this feature proves itself quite powerful already.

With C23 came more ways to define constant data, through \texttt{constexpr}.  Assuming
it ever reaches parity with C++ \texttt{constexpr}, or considering the possible
presence of \href{https://open-std.org/jtc1/sc22/wg14/www/docs/n3600.htm}{N3600} in C2y, this feature could continue to improve over time.
\section{Implementation}
\label{sec:org97de7bb}
Seeing as the more complex C++ ``contract'' system will in any case be implemented
by C++26-compliant compilers, having this feature should not pose a problem to
C2y-compliant compilers. In any case, as was previously stated, calling a function
compiled with \texttt{assume} would still work on previous standards in library-compliant
contexts.

There is seemingly no C compiler extension allowing for this exact feature.  One
could imagine possible function-like macros being able to replicate such a
feature, but it would certainly be non-trivial.

Even then, macros would not be ideal, as they would:
\begin{enumerate}
\item Allow for the library user to call the function without its underlying assumptions,
\item Make compile-time optimisations impossible,
\item Clutter the program with extraneous definitions if we have one macro per function,
\item Be incompatible with style guides wherein parameters are unnamed in declarations,
\item Generally worsen the user experience, as macros are not always well supported
by language servers,
\item Make the assumptions messy and hard to modify; and
\item Come with the usual points of failure of macros (\href{https://wiki.sei.cmu.edu/confluence/display/c/PRE31-C.+Avoid+side+effects+in+arguments+to+unsafe+macros}{CERT-PRE31-C}, notably).
\end{enumerate}

Indeed, it would be much more interesting for it to be a standard feature
instead of being bound by the rules of macros.
\end{document}
