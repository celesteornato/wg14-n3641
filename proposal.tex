% Created 2025-07-04 Fri 10:23
% Intended LaTeX compiler: pdflatex
\documentclass[a4paper, 12pt]{article}
\usepackage[utf8]{inputenc}
\usepackage[T1]{fontenc}
\usepackage{graphicx}
\usepackage{longtable}
\usepackage{wrapfig}
\usepackage{rotating}
\usepackage[normalem]{ulem}
\usepackage{amsmath}
\usepackage{amssymb}
\usepackage{capt-of}
\usepackage{hyperref}
\makeatletter \@ifpackageloaded{geometry}{\geometry{margin=2cm}}{\usepackage[margin=2cm]{geometry}} \makeatother
\author{Céleste Ornato <celeste@ornato.com>}
\date{\today}
\title{Declaration-level static assertions}
\hypersetup{
 pdfauthor={Céleste Ornato <celeste@ornato.com>},
 pdftitle={Declaration-level static assertions},
 pdfkeywords={},
 pdfsubject={},
 pdfcreator={Emacs 30.1 (Org mode 9.7.31)}, 
 pdflang={English}}
\begin{document}

\maketitle
\begin{itemize}
\item \textbf{Proposal category}: Feature request
\item \textbf{Target audience}: Developers writing libraries
\end{itemize}
\section{Abstract}
\label{sec:org5bd3413}
This proposal aims to add the \texttt{[[assume(expr)]]} and \texttt{[[assume(expr, msg)]]}
attributes to function declarations, allowing the designer of a library to specify
statically-verifiable constraints in function parameters.

This features allows both for added safety on the caller side, and possibly for
further optimisation on the compiler side.
\section{Introduction}
\label{sec:orga8228cd}
Inattention and forgetfulness cause developers to write insecure code.

If one wishes to convey to the user of a library that certain function parameters
can cause undefined or unwanted behaviour, they can only do so in written
documentation. This leaves the developer in a precarious position: they can
either add checks at execution-time, which come with an unwanted performance overhead,
or they can assume that everyone will read the documentation, which may create
vulnerabilities.

This proposal seeks to make some of these checks possible without overhead, by implicitly
making static assertions before the function call.

This feature can be seen as an extension to the C99 ``static array index in
function parameters'' feature, in that it both allows for optimisations by the
compiler and for more compile-time checking for the user.
\section{Proposal}
\label{sec:org2a6889a}
\subsection{Technical Description}
\label{sec:org807dffc}
When the attribute \texttt{[[assume(expr)]]} is associated to the declaration of a
function \texttt{foo}, any call to \texttt{foo} will require the compiler to check that \texttt{expr}
is \textbf{not provably false} (see \ref{sec:org1ba83f0}) given the parameters.
When compiling \texttt{foo}, it is assumed that \texttt{expr} is always true. Calling the
function with parameters that do not respect \texttt{expr} is undefined behaviour.

\texttt{expr} may address the function parameters by their name, or by calling them
\texttt{\$n}, where \texttt{n} is the 0-indexed number of the parameter from right to left.

If a referenced variable is an array, it should decay into a pointer. While it
could have allowed for more features, this would have created bug-prone behaviour
depending on whether the author of \texttt{expr} expected a pointer or an array. In any
case, most problems related to the size of arrays are already solved by having
static indices in the function signature.
\subsection{Rationale}
\label{sec:org602c548}
Adding more undefined behaviour in C2y might be controversial, it would certainly
not be immediately seen as ``Enabling secure programming''.

This feature is meant for cases where the developer already considers certain
parameters as ``Undefined behaviour''. At this point, it does not matter
whether the standard considers the code as defined, because the results would
still be unexpected and prone to breakage.

Allowing for further optimisation is just a welcome consequence of one being able
to specify what they consider undefined behaviour. In reality, the wanted feature is the compiler
being capable of detecting mistakes. Function signatures using this attribute also
make the code self-documenting, which makes it easier to know what the developer expected
when writing the function.
\subsection{Example}
\label{sec:org673fae6}
\begin{verbatim}
[[assume($1 != 0)]]
int division(int, int);

[[assume(a >= b, "Function requires a >= b")]]
unsigned int substract(unsigned int a, unsigned int b);

int main(void)
{
    // These compile, with no execution-time overhead.
    int result1 = division(1, 2);
    unsigned int result2 = substract(2, 3);

    // These do not compile.
    int result3 = division(2, 0);           // Error: "Assumption '$1 != 0' is false."
    unsigned int result4 = substract(0, 6); // Error: "Function requires a >= b."
}
\end{verbatim}
\subsection{Quirks}
\label{sec:org1ba83f0}
\textbf{An assumption whose validity cannot be proven will be treated as valid.}
This should not be a problem, as this would just mean coming back to the
status quo of having to be careful as a user.
This is consistent with the way in which static array indices work in
function signatures.

Due to the way static assertions work in general, we cannot always assert
the validity of every expression. However, in ``safer'', more restricted
styles of C, where we only work with automatic-storage-duration variables
and constant-sized arrays, this feature proves itself quite powerful already.

With C23 came more ways to define constant data, through \texttt{constexpr}.
Assuming it ever reaches parity with C++ \texttt{constexpr}, or considering the possible
presence of \href{https://open-std.org/jtc1/sc22/wg14/www/docs/n3600.htm}{N3600} in C2y, this feature could continue to improve over time.
\section{Prior implementations}
\label{sec:org4410275}
There is seemingly no compiler extension allowing for this exact feature.
One could imagine possible function-like macros being able to replicate such
a feature, but it would be non-trivial to write for functions whose return
value isn't discarded.

Even then, macros would not be ideal, as they would:
\begin{enumerate}
\item Allow for the library user to call the function without its underlying assumptions,
\item Make compile-time optimisations impossible,
\item Clutter the program with extraneous definitions if we have one macro per function,
\item Be incompatible with styleguides wherein parameters are unnamed in declarations,
\item Generally worsen the user experience, as macros are not always well supported
by language servers,
\item Make the header files generally messy with hard to modify functions; and
\item Come with the usual points of failure of macros (\href{https://wiki.sei.cmu.edu/confluence/display/c/PRE31-C.+Avoid+side+effects+in+arguments+to+unsafe+macros}{CERT-PRE31-C}, notably).
\end{enumerate}

Indeed, it would be much more interesting for it to be a standard feature
instead of being bound by the rules of macros.
\end{document}
